\chapter{A conclusive comparison}\label{a-conclusive-comparison}

This thesis started with an overview of RP, that introduced the
foundations and the main aspects of the paradigm, and then continued
with a detailed overview of some of the most popular or relevant
frameworks that allow the application of the paradigm in different
platforms and languages, with also some practical use cases.

This final section will attempt to wrap everything up, with a conclusive
comparison of each library peculiarities and the approach to RP that
each library suggests, with also some subjective opinions.

\section{Operators, Expressions,
Declarativness}\label{operators-expressions-declarativness}

In regards to \textbf{expressing a computation in terms of a flow of
intermediate steps}, the approach that Scala.React suggests with its
\textbf{signal expressions} is the main difference in respect to all
other approaches, which are instead based on the presence of
\textbf{operators}. Scala.React's Signal type has no operator at all,
and lets the developer to build a ``reactive'' computation with an
imperative-style code, at least in appearance. In Fact, signal
expressions and the magic behind \texttt{Var} and \texttt{Val} implicit
conversion looks natural to a developer with an object oriented
background.

All the other libraries introduce an approach based on operators,
indeed:

\begin{itemize}
\itemsep1pt\parskip0pt\parsep0pt
\item
  RxJava offers a wide range of \textbf{methods} in the
  \texttt{Observable} type;
\item
  ReactiveCocoa offers a wide range of \textbf{free functions}, used in
  combination with the pipe-forward operator
  \texttt{\textbar{}\textgreater{}} on the \texttt{Signal} and
  \texttt{SignalProducer} type to give the user an elegant way to build
  a chain of operators without losing the purity of the approach based
  on free functions;
\item
  Akka Streams offers a (still) minimal but effective set of
  \textbf{components}, with the fundamental abstraction defined in
  terms of the \texttt{Flow}, \texttt{Sink} and \texttt{Source} type,
  and the ability to build linear or graph computations simply using the
  to \texttt{\textasciitilde{}\textgreater{}} operator.
\end{itemize}

All the libraries previously introduced presented a high level of
declarativness, with a good level of integration in the ecosystem they
belong to:

\begin{itemize}
\itemsep1pt\parskip0pt\parsep0pt
\item
  Scala.React does an amazing job in solving the issues of the
  ``standard'' observer pattern leveraging on the construct offered by
  the Scala language;
\item
  RxJava offer a pretty straight-forward implementation of Rx to the Java
  ecosystem. Both the implementation and the interfaces really
  benefitted by the advent of Java 8 and lambda expression (or, in
  Android, Retrolambda) in the language ecosystem;
\item
  RAC has come a long way from its Objective-C years, and now, with
  Swift and the current 3.0 version, offers a nice and clean set of
  interfaces and abstractions;
\item
  Akka Streams is, in this thesis author's opinion, the most impressive
  approach to modelling computation declaratively. The approach of
  building a graph-resembling DSL to express not-linear computation is a
  feature that really helps in keeping the code clean and intuitive,
  even if the initial learning curve and some of the magic behind the
  library is a thing to always keep in consideration when evaluating a
  tool.
\end{itemize}

\section{Hot and Cold, Push and Pull,
Back-pressure}\label{hot-and-cold-push-and-pull-back-pressure}

Another aspect that distinguishes the libraries is the way they abstract
the notion of \textbf{who actually starts the computation}, and thus,
the propagation of changes.

In this regards, RAC offer a pretty straight forward abstraction putting
this distinction directly at type level: the \texttt{Signal} type is
producer-driven, so the changes are pushed to the subscribers as they
happens/are produced, while the \texttt{SignalProducer} type is demand
driven, so the changes are pulled.

In RxJava this distinction is not obvious, but there're a set of types
and operators that helps when building observables that should be
producer-driven or demand-driven. For example, the \texttt{Subject} type
is usually used to ``warm'' a cold observable into a hot observable.
RxJava is also conforming to Reactive Streams, just like Akka Streams.
Being conform to Reactive Streams means that for both libraries is valid
the principle that a producer can never send more items than a consumer
can handle, using the so called ``dynamic push-pull'' approach.

In practice, the usage of this feature really depends from the use case
to satisfy. For example, for Akka Streams and RxJava the conformation to
Reactive Streams was a pretty straight forward process (engineers from
Netflix, Typesafe, etc.. all find it problematic having a consumer
overwhelmed by a production of items they can handle), since those
frameworks are used to process an high number of items and also to build
infrastructures and services. In the context of mobile applications,
currently RAC doesn't directly solve this issue, since RAC maintainers
believe that figuring out the behavior of side effects is far more
important, since effects dominate GUI applications.

\section{Solving problems the reactive
way}\label{solving-problems-the-reactive-way}

Coming from an OO background, applying RP seems quite unnatural at
first, since the way that most developers learnt programming is
typically with an imperative approach.

What RP propose is a more declarative approach to solving problems, that
leverages to some well defined principles:

\begin{itemize}
\itemsep1pt\parskip0pt\parsep0pt
\item
  a set of minimal abstractions, that allow to wrap computations in
  which latency plays a main role, in an elegant way;
\item
  a set of operators with a clear semantics, that allows to build
  typesafe and statically-typed computation chains, without worrying
  about classic concurrency issues but focusing mainly on better
  expressing what each computation should do.
\end{itemize}

How these principles fits real-world problems is currently a topic of
discussion and this thesis only covered two main contexts of modern
applications.


