\chapter{Future and Promises}\label{future-and-promises}

The appendix about Functional Programming introduced the main patterns
and principles that help when dealing with computations that imply some
effects. Moving a step forward, we can consider also the latency as an
effect.

The following table represents a possible classification of the effects
in programming.

\begin{table}[]
\centering
\caption{The essential effects in programming}
\label{my-label}
\begin{tabular}{|l|l|l|}
\hline
             & {\bf One}     & {\bf Many}        \\ \hline
Synchronous  & T/Try{[}T{]}  & Iterable{[}T{]}   \\ \hline
Asynchronous &  \textbf{Future{[}T{]}} &Observable{[}T{]} \\ \hline
\end{tabular}
\end{table}

This section will introduce a first abstraction to achieve sequential
composition of actions that can take time to complete and that can also
fail: \texttt{Future{[}T{]}}.

\textbf{Future} and \textbf{Promise} are two really powerful
abstractions that can be classified in the \emph{asynchronous
programming} category.

Wikipedia defines Futures and Promises as follows:

\begin{quote}
In computer science, \textbf{future}, \textbf{promise}, and
\textbf{delay} refer to constructs used for synchronization in some
concurrent programming languages. They describe an object that acts as a
proxy for a result that is initially unknown, usually because the
computation of its value is yet incomplete. {[}\ldots{}{]}
\end{quote}

\begin{quote}
A future is a read-only placeholder view of a variable, while a promise
is a writable, single assignment container which sets the value of the
future.
\end{quote}

Futures and Promises promotes an asynchronous and non-blocking approach
to programming. The scale up to use multiple cores on a machine, but
with the currently implementations they don't scale out across nodes.

\section{Futures}\label{futures}

A Future is an immutable handle to a value or a failure that will become
availlable in the future time. Another possible definition is that
Futures are monads that handles \textbf{exceptions} and
\textbf{latency}. A fist simple possible definition in Scala is:

\begin{verbatim}
trait Future[T] {
    def onComplete(callback: Try[T] => Unit)
        (implicit executor: ExecutionContext): Unit
}

object Future {
    def apply(body: => T)
        (implicit context: ExecutionContext): Future[T]
}
\end{verbatim}

This first definition put some important concepts directly in the type
definitions:

\begin{itemize}
\itemsep1pt\parskip0pt\parsep0pt
\item
  a Future is parametrized on a type \texttt{T}
\item
  a Future may \textbf{fail}, returning a \texttt{Try.Failure}
\item
  a Future may \textbf{succeed} (successfully completed), returning a
  \texttt{Try.Success{[}T{]}}
\end{itemize}

Futures are really useful in defining operations that will be executed
at some point in time by some thread (execution context). Futures are
immutable from the developer point of view: once an API returns a
Future, there's is no way to set a value insede a Future. In other
words, a Future may only be assigned once.

A more complete and precise definition for Future is the following:

\begin{verbatim}
trait Awaitable[T] extends AnyRef {
    abstract def ready(atMost: Duration): Unit
    abstract def result(atMost: Duration): T
}

trait Future[T] extends Awaitable[T] {
    def filter(p: T => Boolean): Future[T]
    def flatMap[S](f: T => Future[S]): Future[U]
    def map[S](f: T => S): Future[S]
    def recoverWith(f: PartialFunction[Throwable, Future[T]]): Future[T]
}

object Future {
    def apply[T](body : => T): Future[T]
}
\end{verbatim}

In the scala library, Futures are already implemented in the
\texttt{scala.concurrent} package. In this package,
\texttt{Future{[}T{]}} is a type which denotes future objects, whereas
\texttt{future} is a method which creates and schedules an asynchronous
computation, and then returns a future object which will be completed
with the result of that computation. An example of the usage of Future
coming from the Scala documentation is the following:

\begin{verbatim}
import scala.concurrent._
import ExecutionContext.Implicits.global

val session = socialNetwork.createSessionFor("user", credentials)
val f: Future[List[Friend]] = future {
  session.getFriends()
}
\end{verbatim}

The the previous example shows some key points:

\begin{itemize}
\itemsep1pt\parskip0pt\parsep0pt
\item
  To obtain the list of friends of a user, a request has to be sent over
  a network, which can take a long time. Wrapping the request inside the
  \texttt{future} method \textbf{doesn't block} the rest of the program
  execution while \textbf{waiting} for a response.
\item
  All the computation is performed \textbf{asynchronously}.
\item
  The list of friends becomes available in the future once the server
  responds.
\item
  If the request fails, the future itself will fail.
\end{itemize}

To obtain the value of a Future, the library offers two main solution:

\begin{itemize}
\itemsep1pt\parskip0pt\parsep0pt
\item
  the client \textbf{blocks} its computation and wait until the future
  is completed
\item
  the client \textbf{register a callback} by using the
  \texttt{onComplete} method, that was alredy depicted at the beginning
  of this chapter.
\end{itemize}

The library also provide the companion methods \texttt{onSuccess} and
\texttt{onFailure}, that usually are used alternatively to
\texttt{onComplete}. All these three methods have \texttt{Unit} as the
result type. This is intentional, since the invocations of these methods
cannot be chained.

The presence of these callbacks may lead to what is known as
``asynchronous spaghetti'', where the overall computation is fragmented
into asynchronous handlers. To prevent this possibility, futures provide
combinators which allow a more straightforward composition, such as
\texttt{map}, \texttt{flatMap}, \texttt{filter}, \ldots{}

The \texttt{map} methods takes a future and a mapping function for the
value of the future and produces a new future that is completed with the
mapped value once the original future is successfully completed.

The \texttt{flatMap} method takes a function that maps the value to a
\textbf{new future} g, and then returns a future which is completed once
g is completed.

The \texttt{filter} method creates a new future which contains the value
of the original future only if it satisfies some predicate. Otherwise,
the new future is failed with a \texttt{NoSuchElementException}.

The presence of these methods enable the Future type to also supports
for-comprehension. An example, taken from the documentation, of
for-comprehension in practice is the following:

\begin{verbatim}
val usdQuote = future { connection.getCurrentValue(USD) }
val chfQuote = future { connection.getCurrentValue(CHF) }

val purchase: Future[Int] = for {
  usd <- usdQuote
  chf <- chfQuote
  if isProfitable(usd, chf)
} yield connection.buy(amount, chf)

purchase onSuccess {
  case _ => println("Purchased " + amount + " CHF")
}
\end{verbatim}

that is translated to:

\begin{verbatim}
val purchase: Future[Int] = usdQuote flatMap {
  usd =>
  chfQuote
    .withFilter(chf => isProfitable(usd, chf))
    .map(chf => connection.buy(amount, chf))
}
\end{verbatim}

For-comprehension is a really elegant and concise way to express
\textbf{chain of computations}, leveraging the monadic nature of Futures
to describe the ``\textbf{happy path}'' of the chain and to propagate
(\textbf{materialize}) the exceptions.


\section{Promises}\label{promises}

\textbf{Promises} are an interesting and alternative way to create
futures. A Promise can be seen as a \emph{writable, single-assignment}
container, which completes a Future.

In Scala, a Promise can be represented by the following trait:

\begin{verbatim}
trait Promise[T] {
  def future: Future[T]
  def complete(result: Try[T]): Unit
  def tryComplete(result: Try[T]): Boolean
}

trait Future[T] {
  def onCompleted(f: Try[T] => Unit): Unit
}
\end{verbatim}

A promise \texttt{p} can alternatively complete or fail the future
returned by \texttt{p.future}. In other words, it acts as a
\textbf{mailbox} for the future that it wraps.

The Scala standard library already implements Promise in the
\texttt{scala.concurrent} package.

Promises have a \emph{single-assignment} semantics, so they can be
completed only once and usually are used to implement other operator on
futures.
