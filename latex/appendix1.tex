\chapter{Functional Programming}\label{functional-programming}

As introduced previously, functional programming doesn't allow side
effects and mutable state. Mutable \textbf{state} may be easily
represented by any variable that is not stable or final, and can be
changed or updated inside of an application. A \textbf{mutation} is the
act of updating some state in place.

A simple example that shows how fast the things become
\emph{incidentally complex} is given by the following simple example
from a Justin Spahr-Summers's talk from Github:

\begin{verbatim}
var visible     // → 2 states
var enabled     // → 4 states
var selected    // → 8 states
var highlighted // → 16 states
\end{verbatim}

In other words, state management quickly become hard to maintain and
unpredictable. By avoiding mutable state a programmer has the assurance
that no one else can possibly have changed his application's state, and
can reason about \emph{what a value will be at a given time}.

This chapter will cover just a small introduction of functional domain
modelling and immutable data structures using Scala as the reference
language.

\subsection{Entities and value
objects}\label{entities-and-value-objects}

In functional domain modelling, a system may be represented through the
following main domain elements: entities, value objects and services.

Each \emph{entity} has a unique identity. An entity may have many
attributes, and each attribute may changes in course of the lifetime of
the system. When an attribute within an entity changes, the identity
itself is still the same. In this meta-model, attributes are abstracted
by \emph{value objects}.

A value object is immutable by definition and can be freely shared
across entities. This doesn't mean that an entity's attributes can't
change. To clear this concept, let's consider the following example,
with an entity person within a system, which is univocally identified by
an unique id and that has an ``address'' attribute. Now the difference
between the two concepts is immediate: when we talk of a person we
usually refer to a \emph{specific instance} of person, while when we
talk of an address we just refer to the \emph{value part}.

Functional programming aims to model \textbf{as much immutability as
possible}, so a possible approach to model entities and attributes is
the following:

\begin{itemize}
\itemsep1pt\parskip0pt\parsep0pt
\item
  remember that an \textbf{entity} has an identity that cannot change,
  so it's \textbf{semantically mutable}
\item
  remember that a \textbf{value object} has a value that cannot change,
  so it's \textbf{semantically immutable}
\end{itemize}

The fact that an entity is semantically mutable doesn't prevent us from
using immutable constructs for its implementation.

\subsection{Services}\label{services}

The last element of this meta-model is the \textbf{service}, which is a
more macro level abstraction than entity and value object. In short, a
service models a use case, encapsulating a complete operation that has a
certain value to user of the system.

A service usually involves multiple entities and value objects, that
interact according to specific business rules to deliver specific
functionalities.


\section{Algebraic data types}\label{algebraic-data-types}

From Wikipedia:

\begin{quote}
In computer programming, particularly functional programming and type
theory, an algebraic data type is a kind of composite type, i.e.~a type
formed by combining other types. Two common classes of algebraic type
are product types, i.e.~tuples and records, and sum types, also called
tagged or disjoint unions or variant types.
\end{quote}

The definition introduces some new terminologies: \emph{Sum Type} and
\emph{Product Type}.

Wikipedia defines \textbf{Sum Type} as:

\begin{quote}
a data structure used to hold a value that could take on several
different, but fixed types. Only one of the types can be in use at any
one time, and a tag field explicitly indicates which one is in use.
\end{quote}

An example of a Sum Type in the scala library is Either, that represents
a value of one of two possible types. Instances of Either are either an
instance of Left or Right. A common use of Either is as an alternative
to Option for dealing with possible missing values and convention
dictates that Left is used for failure and Right is used for success.

The implementation of Either is something like:

\begin{verbatim}
sealed abstract class Either[+A, +B]{...}
final case class Left[+A, +B](a: A) extends Either[A, B] {...}
final case class Right[+A, +B](b: B) extends Either[A, B] {...}
\end{verbatim}

Basically, Sum Types express an \emph{OR} of types. Remembering that in
logic OR is presented by a plus, we can say that algebraically:
\emph{type Either = Left + Right}.

Wikipedia defines \textbf{Product Type} as:

\begin{quote}
another, compounded, type in a structure. The ``operands'' of the
product are types, and the structure of a product type is determined by
the fixed order of the operands in the product.
\end{quote}

A Product Type is nothing more than a cartesian product of data types. A
trivial example may be the following:

\begin{verbatim}
sealed trait Animal {
    def uniqueId: Long
    def name: String
    def owner: Option[Person]
}

case class Dog(uniqueId: Long, name: String,
    owner: Option[Person], microchipId: String) extends Animal

case class Hamster(uniqueId: Long, name: String,
    owner: Option[Person]) extends Animal

//...
\end{verbatim}

In this example, a Dog data type is the collection of all valid
combinations of the tuple (Long, String, Option, String). So,
algebraically, we can say: \emph{type Dog = Long x String x Option x
String}.

Always in this example, we also have that Animal is a Sum Type, since an
Animal is a Dog, OR an Hamster, and so on..

Sum Type and Product Type are really important in functional domain
modelling, since they provide the abstraction needed for structuring the
various data of the domain model.

The values of a Sum Type are typically grouped into several classes,
called \emph{variants}. As the name suggests, variants let us model the
variations within a specific data type. Each variant has its own
constructor, which takes a specified number of arguments with specified
types.

Product Types represent a larger abstraction instead, that allows to
\emph{clubbing} together some types and \emph{tagging} it with a new
data type. This extra tag could also be avoided, simply expressing every
Product Type in terms of a tuple of types, but in functional domain
modelling it's always convenient to express an entity with a tagged data
type.

Values of algebraic data types are analyzed with \textbf{pattern
matching}, which helps keep functionality local to the respective
variant of algebraic data type. Pattern matching identifies a value by
its constructor or field names and extracts the data it contains.
Starting from the previous example, a trivial example of pattern
matching:

\begin{verbatim}
  var myAnimal: Animal = ???

  val noise = myAnimal match {
    case Dog(uniqueId, name, owner, microchipId) => "bau!"
    case Hamster(uniqueId, name, owner) => "squit!"
  }
\end{verbatim}

Algebraic data structures, pattern matching and the power of a strongly
typed language like Scala really help in the domain modelling phase.

\section{ADTs and Immutability}\label{adts-and-immutability}

The previous section introduced ADTs as composed types and depicted Sum
Type and Product Type. This section will explain how to achieve
referentially trasparency through \textbf{immutability},
\textbf{avoiding in-place mutations}.

The importance of having immutable data structures is to research in the
fact that they simplify the management of parallel and concurrent
systems. If data structures are immutables, they can be freely shared
between different execution contexts, without any fears.

To better understand what does it mean to have an immutable data
structure, let's consider the following example from the book
``Functional Programming in Scala'':

\begin{verbatim}
// List data type
sealed trait List[+A]

// data constructors
case object Nil extends List[Nothing]
case class Cons[+A](head: A, tail: List[A]) extends List[A]
\end{verbatim}

The operation of addition of an element to the front of the list can be
performed \textbf{without modify or copy} the object itself, just
\textbf{reusing} the actual list with a new element at the beginning.

\begin{verbatim}
  val initialList:List[Int] = Cons(1, Cons(2, Nil))
  val myList = Cons(0, initialList)
\end{verbatim}

This approach can be obviously used also to remove an element from the
list (in the following example, the 1 to the front).

\begin{verbatim}
  val initialList:List[Int] = Cons(1, Cons(2, Nil))

  val myList1 = initialList match {
    case Cons(1, xs) => xs
    case l: List[Int] => l
  }
\end{verbatim}

This property of reuse object instead of modifying or copying the object
itself is called \emph{sharing}.

The introduction of this chapter depicted three elements as the
fundamental domain elements: entities, value objects and services. So,
how do ADTs relate to these elements? As previously written, a value
object is \emph{semantically immutable}, so abstracting this with an ADT
is a good choice. For example, let's consider the following example:

\begin{Shaded}
\begin{Highlighting}[]
\KeywordTok{case} \KeywordTok{class} \FunctionTok{Address}\NormalTok{(no: String, street: String, city: String, state: String, zip: String)}
\end{Highlighting}
\end{Shaded}

If an application, at a given time, has an instance of an address and
needs to modify an attribute (e.g.~the zip code), all it has to do is to
\textbf{generate a new instance} of the object with the updated
attribute, \textbf{avoiding in place mutation} (no vars, no setters).
Scala also offer a \texttt{copy} method to further simplify the job.

\begin{verbatim}
val initialAddress = Address("10", "Via Sacchi", "Cesena", "IT", "47522")
val newAddress = initialAddress.copy(zip = "47523")
\end{verbatim}

This approach may looks nice at a first sight, but doesn't scale well,
unfortunately. To see the problem, just consider a new entity that uses
the address as a value object.

\begin{verbatim}
case class Person(id: Long, name: String, address: Address)
\end{verbatim}

The entity Person is \emph{semantically mutable}, but it's implemented
with an ADT. The first part of this chapter already stated the fact that
an entity is semantically mutable doesn't prevent us from using
immutable constructs for its implementation, and this is a brilliant
proof of concept.

Back to the issues with when using \texttt{copy} for creating a new ADTs
with modified fields, let's consider the following simple example.

\begin{verbatim}
val initPerson = Person(0, "Alessandro", initialAddress)
val newPerson = initPerson.copy(address = initPerson.address.copy(zip = "47523"))
\end{verbatim}

It immediately appears that the code is getting bad. And things only get
worse when there are multiple level of nesting of the objects. A better
abstraction to solve this problem is given by \emph{Lenses}. Lenses are
ADTs, and in Scala can be implemented as follows:

\begin{verbatim}
case class Lens[O, V](
    get: O => V,
    set: (O, V) => O
)
\end{verbatim}

From the implementation it emerges that \textbf{Lenses}:

\begin{itemize}
\itemsep1pt\parskip0pt\parsep0pt
\item
  are \textbf{parametrized} (on O and V types)
\item
  have a \textbf{getter} method, which takes an object with type O and
  return a value of type V
\item
  have a \textbf{setter} method, which takes an object of type O and
  returns a new instance of the object set to the value
\end{itemize}

To demonstrate that lenses just work as the copy method introduced below,
let's consider the following example:

\begin{verbatim}
val newAddress2 = addressZipLens.set(initialAddress, "47523")
newAddress == newAddress2 //> res0: Boolean = true
\end{verbatim}

The compare on the final line asserts that the addresses created with
the two different techniques are equals.

To solve the problem related to the multiple level of nesting attribute
update it is necessary to introduce a generic \texttt{compose} function,
implemented as follows:

\begin{verbatim}
def compose[Outer, Inner, Value](
    outer: Lens[Outer, Inner],
    inner: Lens[Inner, Value]
) = Lens[Outer, Value](
    get = outer.get andThen inner.get,
    set = (obj, value) => outer.set(obj, inner.set(outer.get(obj), value)))
\end{verbatim}

Compose, as the name suggests, takes care of composing two lenses. Lens
composition is really helpful to maintain the code clean and safe, as the
following sample will demonstrate.

\begin{verbatim}
val personAddressZipLens: Lens[Person,String] 
	= compose(personAddressLens, addressZipLens)
newPerson2 = personAddressZipLens.set(initPerson, "47523")
newPerson == newPerson2 //> res1: Boolean = true
\end{verbatim}

As the previous example, also in this case the result is the same as the
case when \texttt{copy} was used instead. Lenses offer a \emph{view on
data}, allowing to get and modify the data \emph{the functional way}.

\section{Referential trasparency}\label{referential-trasparency}

An important concept in functional programming is referential
transparency. Referential transparency is a property of programs, which
makes it easier to verify, optimize, and parallelize programs.

From Wikipedia's definition:

\begin{quote}
\textbf{Referential transparency} and \textbf{referential opacity} are
properties of parts of computer programs. An expression is said to be
referentially transparent if it can be replaced with its value without
changing the behavior of a program (in other words, yielding a program
that has the same effects and output on the same input). The opposite
term is referential opaqueness.
\end{quote}

A necessary, but not sufficient, condition for referential transparency
is the absence of side effects. In simple words, it's all about the fact
that an expression can be replaced with its value. This obviously
requires that the expression (that could be a function call, for
example) has no side effects and always return the same output on the
same input.

In other words, referential transparency means that the value of
expression can depend only on the values of its parts, and not on any
other facts about them or about the execution context.

The opposite of referential transparency is referential opacity, where
the value does change when evaluated, and the expression cannot be
replaced by a single value without altering the way a program executes.

A trivial but effective example of referential transparency has been
introduced in the previous section, with the code snippet about lists.
Using immutable lists, the act of adding, removing or updating a value
in those data structures results in new lists being created with the
changed values, and any part of the program still observing the original
lists sees no change.

\subsection{Equational reasoning}\label{equational-reasoning}

The previous section introduced and depicted referential transparency.
The importance of referential transparency expressions is that they make
substitution model work and the substitution model helps
\textbf{equational reasoning} work.

A possible definition of equational reasoning is \emph{the process of
interpreting code by substituting equals-for-equals}.

Equational reasoning is an important property, since it keeps the models
easy to reason about and easy to validate. The correctness of the
properties of a model that respect equational reasoning can be verified
just as the properties of a math theorem.


\section{Patterns}\label{patterns}

The previous sections of this chapter introduced the basics of
functional domain modelling, by using \textbf{algebraic data types}, and
the importance of \textbf{referential transparency}.

This final section will go deeper, and cover three functional design
pattern: \textbf{monoids}, \textbf{functors} and \textbf{monads}. These
patterns are really important to fully understand the usage of algebraic
data type to model the domain.

The pattern introduced in this section are \emph{abstract}, meaning that
they don't operate on just one specific data type bar on all data types
that share a common algebra.

\subsection{Monoid}\label{monoid}

\textbf{Monoid} is an ubiquitous pattern that come up frequently in
programming, even if we are not aware of it. To introduce monoids, let's
consider an example about the concatenation operation for List and
String types:

\begin{itemize}
\itemsep1pt\parskip0pt\parsep0pt
\item
  the concatenation operation is a binary operation and is associative
\item
  both List and String have an identity element and the operation
  applied with the identity yields the same value as the other argument
\end{itemize}

The points depicted above are a valid algebra for the abstraction of a
monoid. In Scala a monoid can be expressed as follows.

\begin{verbatim}
trait Monoid[A] {
  def op(a1: A, a2: A): A
  def zero: A
}
\end{verbatim}

More formally, a monoid needs to satisfy the following laws:

\begin{itemize}
\itemsep1pt\parskip0pt\parsep0pt
\item
  \emph{Left identity}: \texttt{op(zero,\ a)\ ==\ a}
\item
  \emph{Right identity}: \texttt{op(a,\ zero)\ ==\ a}
\item
  \emph{Associativity}:
  \texttt{op(a1,\ op(a2,\ a3))\ ==\ op(op(a1,\ a2),\ a3)}
\end{itemize}

Back to the initial example, with the given definition List and String
concatenation can be expressed as monoids as follows.

\begin{verbatim}
val stringMonoid = new Monoid[String] {
  def op(a1: String, a2: String) = a1 + a2
  def zero = ""
}

def listMonoid[A] = new Monoid[List[A]] {
  def op(a1: List[A], a2: List[A]) = a1 ++ a2
  def zero = Nil
}
\end{verbatim}

The benefits of having operations that operate on different data types
modelled as monoids is given by the fact that monoids offer
\emph{parametricity} and the ability to \emph{abstract over behaviors}.

A brilliant example that illustrates the power of monoids is given when
they are used in conjunction with lists and list folding functions.
Looking at \texttt{foldLeft} function definition:

\begin{verbatim}
def foldLeft[B](z: B)(f: (B, A) => B): B
\end{verbatim}

and in the particular case of \texttt{A\ =\ B}:

\begin{verbatim}
def foldLeft(z: A)(f: (A, A) => A): A
\end{verbatim}

it can be observed that the function signature perfectly matches the
signature of \texttt{op} operation on the monoid type.

This brings to a really nice proof of concept about the practical usage
of monoids, like in the following example.

\begin{verbatim}
val words = List("Hello", "world")
val t = words.foldLeft(stringMonoid.zero)(stringMonoid.op)
\end{verbatim}

\subsection{Functor}\label{functor}

The previous section introduced an ubiquitous pattern that recurs
frequently in programming, monoid. This section will make a further step
towards the king abstraction of functional programming, which will be
introduced in the next one.

In the Scala standard library there are some classes that provide a
\texttt{map} function. For example:

\begin{verbatim}
def map[B](f: (A) => B): List[B]

def map[B](f: (A) => B): Option[B]
\end{verbatim}

In the classes of the example, the effect of \texttt{map} is:

\begin{itemize}
\itemsep1pt\parskip0pt\parsep0pt
\item
  for List, \texttt{map} applies the function \texttt{f} on each element
  of the list
\item
  for Option, \texttt{map} applies the function \texttt{f} only if the
  element is instance of \texttt{Some} and return \texttt{Some} of the
  result, otherwise it returns \texttt{None}
\end{itemize}

All these function signatures are pretty the same, with the only
difference of the concrete type involved. As always in programming, when
there are things that are pretty the same, it's possible that factorize
the behavior and create a new abstraction. In this case, the abstraction
it's all about a \emph{data type that implements map} and it's called
\textbf{functor}. In Scala, if a data type \texttt{F} implements
\texttt{map}, \texttt{F} is a functor. An implementation of this
abstraction can be represented by the following trait.

\begin{verbatim}
trait Functor[F[_]] {
    def map[A, B](fa: F[A])(f: A => B): F[B]
}
\end{verbatim}

\textbf{NB}: in the example provided above (from the Scala library), the
signatures of the map function are different from the signature of map
in the trait introduced below. The reason of this difference is due to
the object oriented nature of the classes in the Scala library. The new
signature has a more ``functional'' approach, with an additional
parameter (the first one) that takes the object to map on.

With this new abstraction defined, it's possible to define functors
explicitly for List and Option in the following way, reusing the object
oriented implementation provided by the Scala standard library:

\begin{verbatim}
def listFunctor: Functor[List] = new Functor[List] {
    def map[A, B](a: List[A])(f: A => B): List[B] = a map f
}

def optionFunctor: Functor[Option] = new Functor[Option] {
    def map[A, B](a: Option[A])(f: A => B): Option[B] 
    	= a.map(f).orElse(None)
}
\end{verbatim}

Functors are nothing more that an abstraction that has the capability of
mapping over some data structure, with functions.

\subsection{Monad}\label{monad}

The concept of monad comes from category theory, and is one of the most
important topics in functional programming development.

From Wikipedia:

\begin{quote}
In functional programming, a \textbf{monad} is a structure that
represents computations defined as sequences of steps: a type with a
monad structure defines what it means to chain operations, or nest
functions of that type together. This allows the programmer to build
pipelines that process data in steps, in which each action is decorated
with additional processing rules provided by the monad. As such, monads
have been described as ``programmable semicolons''; a semicolon is the
operator used to chain together individual statements in many imperative
programming languages, thus the expression implies that extra code will
be executed between the statements in the pipeline.
\end{quote}

From a Erik Meijer's quote:

\begin{quote}
Monads are return types that guide you through the happy path.
\end{quote}

From a Martin Odersky's quote:

\begin{quote}
Monads are parametric types with two operations flatMap and unit that
obey some algebraic laws.
\end{quote}

An other definition can be: \textgreater{}Monads are abstract
computations that help us mimic the effects of typically impure actions
like exceptions, IO, continuations etc. while providing a functional
interface to the users.

All this quotes and definitions help to give us the idea of what a monad
is. Formally, a monad is just a type, and can be defined as follows:

\begin{verbatim}
trait Monad[M[_]] extends Functor[M] {
  def unit[A](a: => A): M[A]
  def flatMap[A,B](ma: M[A])(f: A => M[B]): M[B]

  def map[A,B](ma: M[A])(f: A => B): M[B] =
    flatMap(ma)(a => unit(f(a)))
}
\end{verbatim}

Monad extends functor, implementing \texttt{map}. The trait introduced
above is generic, so each type that implements a \texttt{unit} and a
\texttt{flatMap} and respects some laws (depicted below) is a monad.

\textbf{NB}: in the literature, \texttt{flatMap} is often called
\texttt{bind} and \texttt{unit} is often called \texttt{return}.

Two examples of monads are represented by \texttt{Option} and
\texttt{List}. Starting from the definition of Monad, they can be
defined as follows (reusing the implementation provided by the Scala
standard library):

\begin{verbatim}
val optionMonad = new Monad[Option] {
    def unit[A](a: => A) = Some(a)
    def flatMap[A,B](ma: Option[A])(
    	f: A => Option[B]) = ma flatMap f
}

val listMonad = new Monad[List] {
    def unit[A](a: => A) = List(a)
    def flatMap[A,B](ma: List[A])(f: A => List[B]) = ma flatMap f
}
\end{verbatim}

From the example, it's easy to note that \texttt{unit} is different for
each monad. For example, for the \texttt{List} type unit is
\texttt{List(a)} and for the \texttt{Option} type unit is
\texttt{Some(a)}.

A more object oriented definition for monad is the following.

\begin{verbatim}
trait M[T] {
    def flatMap[U](f: T => M[U]): M[U]
    def unit[T](x: T): M[T]
}
\end{verbatim}

with \texttt{map} defined as
\texttt{m\ flatMap\ (x\ =\textgreater{}\ unit(f(x)))}.

As introduced previously, monads need to satisfy three laws:
associativity, left unit and right unit.

\subsubsection{Associativity}\label{associativity}

The associativity law is about the order of \textbf{composition}, and
demands that for any monad \texttt{m}and any two functions \texttt{f}
and \texttt{g}, it must not make a difference whether you apply f to m
first and then apply g to the result, or if you first apply g to the
result of f and then in turn flatMap this over m. Formally:

\begin{verbatim}
m flatMap f flatMap g == m flatMap (x => f(x) flatMap g)
\end{verbatim}

\subsubsection{Left unit}\label{left-unit}

The left unit law demands that flatMap must behave in such a way that
for any function f passed to it, the result is the same as calling f in
\emph{isolation}. Formally:

\begin{verbatim}
unit(x) flatMap f  ==  f(x)
\end{verbatim}

The left unit law can be considered the most important law, since it
guarantees that flatMap let's you apply a transformation to the value
contained in a monad, without leaving the monad. In other words, monads
promote and allow \emph{containment} and \emph{chainability} of
transformations.

\subsubsection{Right unit}\label{right-unit}

The right unit law demands that applying unit to a monad has no effects
at all (or, better, has the same outcome as not calling it at all).
Formally:

\begin{verbatim}
m flatMap unit  ==  m
\end{verbatim}

Usually the term monad is also used \emph{informally}, to describe a
type that has \texttt{flatMap} and \texttt{unit}, without any attention
about the monad laws. An example of this is given by the \texttt{Try}
type, which has a \texttt{Success} case that contains a value and a
\texttt{Failure} case that contains an exception.

\begin{verbatim}
abstract class Try[+T]
case class Success[T](x: T)       extends Try[T]
case class Failure(ex: Exception) extends Try[Nothing]
\end{verbatim}

\textbf{Try} is often used to wrap the result of a computation inside a
container type. The important thing about Try is that is handles
exception, through \emph{materialization}.

\begin{verbatim}
def map[S](f: T => 􏰀S): Try[S] = this match {
  case Success(value)       => Try(f(value))
  case failure: Failure(t)  => failure
}


def flatMap[S](f: T => 􏰀Try[S]): Try[S] = this match {
  case Success(value)       =>
    try { f(value) } catch { case NonFatal(t) => Failure(t) }
  case failure: Failure(t)  => failure
}
\end{verbatim}

At first sight Try looks like a monad, implementing \texttt{flatMap}
(and \texttt{unit}), but formally it doesn't respect the left unit law.


